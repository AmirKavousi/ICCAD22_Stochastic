\documentclass[conference]{IEEEtran}
\IEEEoverridecommandlockouts
% The preceding line is only needed to identify funding in the first footnote. If that is unneeded, please comment it out.
%\usepackage{widetext}
\usepackage{mathtools}


\usepackage{cite}
\usepackage{amsmath,amssymb,amsfonts}
\usepackage{algorithmic}
\usepackage{graphicx}
\graphicspath{ {./figures/} }
\usepackage{subfigure}
\usepackage{multirow}
\usepackage{lipsum}
\usepackage{cuted}
\usepackage{textcomp}
\usepackage{diffcoeff}
\usepackage{xcolor}
\usepackage[outdir=./]{epstopdf}
\usepackage[ruled,vlined]{algorithm2e}
\def\BibTeX{{\rm B\kern-.05em{\sc i\kern-.025em b}\kern-.08em
    T\kern-.1667em\lower.7ex\hbox{E}\kern-.125emX}}

\newtheorem{proposition}{Proposition}

\renewcommand{\baselinestretch}{0.83}

\begin{document}



\title{Coupled Electromigration and IR Drop Analysis Considering Spatial Joule Heating
Effects}

% \author{\IEEEauthorblockN{Mohammadamir Kavousi, Liang Chen and Sheldon X.-D. Tan}
% \IEEEauthorblockA{Department of Electrical and Computer Engineering, University of California, Riverside, CA 92521 USA\\
% mkavo003@ucr.edu, lianchen@ucr.edu, stan@ece.ucr.edu}
% }

\maketitle
\raggedbottom
\begin{abstract}
  
  Temperature gradient due to Joule heating has huge impacts on the
  electromigration (EM) induced failure effects. However, Joule
  heating and related thermomigration (TM) effects were less
  investigated in the past for physics-based EM analysis for VLSI chip
  design. In this work, we propose EM simulation tool called EMSpice2
  simulator which is based on Emspice simulator[??]. EMSpice does not consider temperature gradient impacts. However all steps in EMSpice2,
  from immortality checks to hydrostatic stress solver, do consider thermomigration effects. In addition EMSpce2 is equipped with MOR to speed up the simulation process by 60x. Moreover in EMSpice2, a novel post-voiding junction model
  considering joule heating effect is proposed.  In existing model, we have to
  partition circuit to sub-circuits to calculate post-voiding phase,
  however in proposed method there is no need to change the topology
  of circuit.
EMSpice2, similar to EMSpice, starts from first principles and simultaneously considers two major interplaying physics effects in EM failure process: the hydrostatic stress and electronic current/voltage in a power grid network. this tool starts by reading the power grid layout information from
Synopsys IC Compiler. It then removes immortal interconnect
wires by considering both temperature-aware nucleation phase immortality and temperature-aware
incubation phase immortality for multi-segment interconnects.
Thereafter, a temperature-aware finite difference time domain (FDTD) solver is
employed for stress analysis for every mortal interconnect tree
for both nucleation and post-voiding phases. For post-voiding phase, even if voids nucleats internal junction, EMSpice2 does not change the topology of circuits. At the whole power
grid circuit level, the EM analysis is coupled with IR drop
analysis of a whole power grid network at each time step so
that we can consider the interaction among stress, void growth,
resistance change and IR drop in a single simulation framework.
Accuracy of EMSpice2 is validated by comparing with finite
element method based COMSOL for nucleation and post-voiding phases.
The numerical results show
that, compared to the recently proposed EMSpice
simulator, EMSpice2 tool which consider TM effects is more accurate and can lead to about 60x speedup on
average for the interconnects for both
void nucleation and growth phases.
  

\end{abstract}

\input intro.tex

\input stochastic.tex

\input results_iccad22.tex

\section{Conclusion}
\label{sec:conclusion}
In this article, we have proposed a new post voiding junction model for
transient EM stress analysis method for multi-segment
interconnects.


\bibliographystyle{unsrt}
%\footnotesize
\bibliography{../../bib/softerror,../../bib/reinforcement,../../bib/stochastic,../../bib/simulation,../../bib/modeling,../../bib/reduction,../../bib/misc,../../bib/architecture,../../bib/mscad_pub,../../bib/thermal_power,../../bib/thermal,../../bib/reliability,../../bib/embedded,../../bib/machine_learning,../../bib/physical,../../bib/neural_network.bib}

\end{document}
