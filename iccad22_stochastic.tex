\documentclass[conference]{IEEEtran}
\IEEEoverridecommandlockouts
% The preceding line is only needed to identify funding in the first footnote. If that is unneeded, please comment it out.
%\usepackage{widetext}
\usepackage{mathtools}


\usepackage{cite}
\usepackage{amsmath,amssymb,amsfonts}
\usepackage{algorithmic}
\usepackage{graphicx}
\graphicspath{ {./figures/} }
\usepackage{subfigure}
\usepackage{multirow}
\usepackage{lipsum}
\usepackage{cuted}
\usepackage{textcomp}
\usepackage{diffcoeff}
\usepackage{xcolor}
\usepackage[outdir=./]{epstopdf}
\usepackage[ruled,vlined]{algorithm2e}
\def\BibTeX{{\rm B\kern-.05em{\sc i\kern-.025em b}\kern-.08em
    T\kern-.1667em\lower.7ex\hbox{E}\kern-.125emX}}

\newtheorem{proposition}{Proposition}

\renewcommand{\baselinestretch}{0.83}

\begin{document}



\title{Electromigration Analysis Using a Stochastic Current \& Activation Energy Considering Spatial Joule Heating
Effects}

% \author{\IEEEauthorblockN{Mohammadamir Kavousi, Liang Chen and Sheldon X.-D. Tan}
% \IEEEauthorblockA{Department of Electrical and Computer Engineering, University of California, Riverside, CA 92521 USA\\
% mkavo003@ucr.edu, lianchen@ucr.edu, stan@ece.ucr.edu}
% }

\maketitle
\raggedbottom
\begin{abstract}
  
  Temperature gradient due to Joule heating has huge impacts on the
  electromigration (EM) induced failure effects. However, Joule
  heating and related thermomigration (TM) effects were less
  investigated in the past for physics-based EM analysis for VLSI chip
  design. In this work, we propose EM simulation tool called EMSpice2
  simulator which is based on Emspice simulator[??]. EMSpice does not consider temperature gradient impacts. However all steps in EMSpice2,
  from immortality checks to hydrostatic stress solver, do consider thermomigration effects. In addition EMSpce2 is equipped with MOR to speed up the simulation process by 60x. Moreover in EMSpice2, a novel post-voiding junction model
  considering joule heating effect is proposed.  In existing model, we have to
  partition circuit to sub-circuits to calculate post-voiding phase,
  however in proposed method there is no need to change the topology
  of circuit.
Electromigration (EM) degradation evolves slowly towards failure,
over a period of years. This is why EM checking methods use effective
current models to represent the underlying circuit workload,
which are typically constant (DC) currents over time. However,
ignoring all input current variations around the mean can be
risky, because low-frequency input variations can have a significant
impact on EM, resulting in shorter than expected lifetimes.
With the use of dark silicon and multimodal chip operation, such
low-frequency changes inworkload are becoming increasingly common
in modern designs. Ignoring these variations can lead to false
positives and must be avoided. We tackle this by developing a stochastic
effective current model for the input current waveforms
that is easy for users to specify and which allows stochastic estimation
of the impact of input variability on the lifetime.

\end{abstract}

\input intro.tex

\input stochastic.tex

\input results_iccad22.tex

\section{Conclusion}
\label{sec:conclusion}
In this article, we have proposed a new post voiding junction model for
transient EM stress analysis method for multi-segment
interconnects.


\bibliographystyle{unsrt}
%\footnotesize
\bibliography{../../bib/softerror,../../bib/reinforcement,../../bib/stochastic,../../bib/simulation,../../bib/modeling,../../bib/reduction,../../bib/misc,../../bib/architecture,../../bib/mscad_pub,../../bib/thermal_power,../../bib/thermal,../../bib/reliability,../../bib/embedded,../../bib/machine_learning,../../bib/physical,../../bib/neural_network.bib}

\end{document}
